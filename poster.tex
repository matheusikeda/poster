\documentclass{sciposter}

\usepackage[portuguese]{babel}
\usepackage[utf8]{inputenc}
\usepackage[centertags]{amsmath}
\usepackage{geometry}
\geometry{paperwidth=90cm,paperheight=120cm,centering, textwidth=77cm, textheight=87cm, left=3cm, top=3cm}
%\newfont{20pt}
\usepackage{color}
\usepackage{amsmath,amsfonts,amssymb,amsthm}
\usepackage{graphicx,colortbl}
\usepackage{lipsum} %sugest\~ao de texto
\usepackage{proof}
\usepackage{listings} % exibe codigos de linguagens
\usepackage{showexpl} % exibe codigos de figuras


\usepackage{multicol}

%include lhs2TeX.fmt
%include lhs2TeX.sty
%include polycode.fmt

\DeclareMathAlphabet{\mathkw}{OT1}{cmss}{bx}{n}
%subst keyword a = "\K{" a "}"
%subst conid a = "\C{" a "}"
%subst varid a = "\V{" a "}"
%subst numeral a = "\N{" a "}"
%format Set = "\D{Set}"
%format Nat = "\C{\mathbb{N}}"


\definecolor{BoxCol}{RGB}{150,32,56}
\definecolor{Verde}{RGB}{50,180,50}
\newcommand{\nat}{\mathbb{N}}

\newcommand{\cod}{\ttfamily\small}


\newcommand{\tituloA}[1]{\Large{\emph{\textbf{\color{white}{#1}}}}}
\newcommand{\tituloB}[1]{\Large{\emph{\textbf{\color{black}{#1}}}}}
\renewcommand{\thesection}{\textcolor{white}{\arabic{section}}}
\renewcommand{\thesubsection}{\textcolor{white}{\arabic{section}.\arabic{subsection}}}
\newcommand{\R}{\mathbb{R}}

\usepackage{color}
\newcommand{\redFG}[1]{\textcolor[rgb]{0.6,0,0}{#1}}
\newcommand{\greenFG}[1]{\textcolor[rgb]{0,0.4,0}{#1}}
\newcommand{\blueFG}[1]{\textcolor[rgb]{0,0,0.8}{#1}}
\newcommand{\orangeFG}[1]{\textcolor[rgb]{0.8,0.4,0}{#1}}
\newcommand{\purpleFG}[1]{\textcolor[rgb]{0.4,0,0.4}{#1}}
\newcommand{\yellowFG}[1]{\textcolor{yellow}{#1}}
\newcommand{\brownFG}[1]{\textcolor[rgb]{0.5,0.2,0.2}{#1}}
\newcommand{\blackFG}[1]{\textcolor[rgb]{0,0,0}{#1}}
\newcommand{\whiteFG}[1]{\textcolor[rgb]{1,1,1}{#1}}
\newcommand{\yellowBG}[1]{\colorbox[rgb]{1,1,0.2}{#1}}
\newcommand{\brownBG}[1]{\colorbox[rgb]{1.0,0.7,0.4}{#1}}

\newcommand{\ColourStuff}{
  \newcommand{\red}{\redFG}
  \newcommand{\green}{\greenFG}
  \newcommand{\blue}{\blueFG}
  \newcommand{\orange}{\orangeFG}
  \newcommand{\purple}{\purpleFG}
  \newcommand{\yellow}{\yellowFG}
  \newcommand{\brown}{\brownFG}
  \newcommand{\black}{\blackFG}
  \newcommand{\white}{\whiteFG}
}

\ColourStuff

\newcommand{\D}[1]{\blue{\mathsf{#1}}}
\newcommand{\C}[1]{\green{\mathsf{#1}}}
\newcommand{\F}[1]{\green{\mathsf{#1}}}
\newcommand{\V}[1]{\blue{\mathit{#1}}}
\newcommand{\N}[1]{\purple{\mathit{#1}}}
\newcommand{\K}[1]{\orange{\mathkw{#1}}}

\newtheorem{theorem}{Teorema}
\newtheorem{lemma}{Lema}
\newtheorem{corollary}{Corolário}

\begin{document}
\colorbox{BoxCol}{
    \begin{minipage}{\textwidth}
      \color{white}{
        \begin{center}
          \huge{\textbf{\vspace{1cm} \\
  Uso de lógica temporal linear para teste baseado em propriedades de sistemas embarcados 
          \vspace{1cm} \\}}
    \end{center}
  }
\end{minipage}
}
\title{} % n\~ao delete
\author{\textbf{Matheus Takeshi Yamakawa Ikeda;Rodrigo Geraldo Ribeiro}}
\institute{Universidade Federal de Ouro Preto - Campus Jo\~ao Monlevade}
\email{\texttt{\{matheusikeda10;rodrigogribeiro\}@gmail.com}}

\leftlogo[0.6]{logo.jpg} %logotipo da universidade
%\rightlogo[1.3]{logoICEA.jpeg}
%\nologos
\qquad %espa\c{c}o
\maketitle %gera titulo


	\colorbox{BoxCol}{
	  \begin{minipage}{\textwidth}
	  	\color{white}{
	    \begin{center}
	   	  \huge{\textbf{\vspace{1cm} \\
	  		 \Large{Encontro de Saberes{}}
	   	  \vspace{1cm} \\}}
	    \end{center}
	}
	  \end{minipage}
	}

	\quad

  \begin{multicols}{3}{

  %Paragrafo.
  \setlength{\parindent}{2em}

  \section*{\tituloA{Introdução}}
    \Large Teste é a abordagem mais utilizada por desenvolvedores de todo o mundo para a garantia da qualidade de software. Uma abordagem inovadora para atividades de teste é o chamado teste baseado em propriedades, em que valores para teste são produzidos de forma automática e verificados com respeito a propriedades descritas como fórmulas da lógica de primeira ordem. 
    \par Testar sistemas embarcados é uma tarefa complexa, sendo muitas vezes feita com o sistema em produção, o que pode acarretar o aumento do custo do desenvolvimento do software, além de não serem facilmente especificados utilizando fórmulas da lógica de primeira ordem. 
    
\section*{\tituloA{Objetivos}}
O principal objetivo desse projeto é a especificação e implementação de uma biblioteca para teste baseado em propriedades da lógica temporal para o teste de sistemas embarcados usando a linguagem Haskell.

\section*{\tituloA{Sintaxe da Lógica Temporal}}

A lógica temporal é a base para a verificação, e neste projeto, usamos a LTL. Esta lógica permite fazer referência ao futuro e modela o tempo como uma sequência de estados.
\par A sintaxe é dada como:
\begin{itemize}
	 \item Uma fórmula da LTL pode ser verdadeira ou falsa.
	 \item Todos os átomos em algum conjunto Átomos são fórmulas da LTL.
	 \item $\neg \phi$ é uma fórmula da LTL se $\phi$ for uma.
	 \item Conectivos temporais: X (próximo estado), F (algum estado futuro), G (todos os estados futuros), U (até), R (solto) e W (fraco).
\end{itemize}

\section*{\tituloA{Semântica da Lógica Temporal}}

Um sistema de transição é modelado através de estados(estrutura estática) e transições(estrutura dinâmica). Um modelo possui um conjunto de estados, relações de transição e um conjunto de proposições atômicas.
\par A semântica dos operadores temporais é definida a partir de uma trajetória específica.
\begin{itemize}
	\item X $\phi$ ($\phi$ será válida no próximo estado)
	\item F $\phi$ (a fórmula $\phi$ será válida em algum estado futuro)
	\item G $\phi$ ($\phi$ sempre será válida)
	\item $\phi _{1}$ U $\phi _{2}$ ($\phi _{1}$ é válida até que $\phi _{2}$ seja)
	\item $\phi _{1}$ R $\phi _{2}$ ($\phi _{2}$ tem que permanecer verdadeira até, e incluindo o momento em que $\phi _{1}$ se torne verdadeira)
	\item $\phi _{1}$ W $\phi _{2}$ (como o conectivo U, porém não exige que $\phi _{2}$ tenha que ser válida, finalmente ao longo da trajetória) 
\end{itemize}

\section*{\tituloA{Resultados Obtidos}}
O projeto teve como resultado a implementação de um teste de satisfazibilidade de fórmulas da LTL na linguagem funcional. Basicamente, uma função recursiva e um novo tipo de dados nomeado LTL.\\
checkLTL :: Interp a -$>$ LTL -$>$ [Env a] -$>$ Bool	

\section*{\tituloA{Conclusões}}
Através de estudos sobre as bibliotecas, foi possível realizar o teste de satisfazibilidade, mas não todo os objetivos propostos inicialmente dentro do período de vigência do projeto. As atividades finalizaram-se antes da implementação em um sistema embarcado, porém até o momento de estudo, pudemos adquirir muitos conhecimentos nunca estudados anteriormente.

  \section*{\tituloA{Referências}}
  \renewcommand{\section}[2]{}%
  \bibliographystyle{plain}
  \bibliography{bib}


  }\end{multicols}



\end{document}
